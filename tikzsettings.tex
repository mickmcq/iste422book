%\usepackage{pgf-umlcd}
% above wreaks havoc with lualatex but not xelatex
\usepackage{tikz}
\usetikzlibrary{automata}
\usetikzlibrary{arrows}
\usetikzlibrary{backgrounds}
\usetikzlibrary{decorations.text}
\usetikzlibrary{decorations.markings}
\usetikzlibrary{fit}
\usetikzlibrary{graphs}
\usetikzlibrary{shapes.geometric}
\pgfdeclarelayer{edgelayer}
\pgfdeclarelayer{nodelayer}
\pgfdeclarelayer{connectionlayers}
\pgfsetlayers{background,edgelayer,nodelayer,connectionlayers,main}
    \tikzstyle{surround} = [
      fill=none,
      thick,
      draw=none,
      rounded corners=2mm
    ]
    \tikzstyle{entity} = [
      align=center,
      fill=entitycolor,
      anchor=base,
      text=white,
      shape=rectangle,
      opacity=0.95,
      draw=none,
      minimum size=3pt,
      inner sep=3pt,
      font=\sffamily\slshape\small
    ]
    \tikzstyle{process} = [
      align=center,
      fill=processcolor,
      anchor=base,
      text=white,
      shape=ellipse,
      minimum size=3pt,
      opacity=0.95,
      inner sep=2pt,
      font=\sffamily\small\slshape
    ]
    \tikzstyle{chenattr} = [
      align=center,
      draw=chenattrcolor,
      thick,
      anchor=base,
      text=chenattrcolor,
      shape=ellipse,
      minimum size=3pt,
      opacity=0.95,
      inner sep=2pt,
      font=\ttfamily\small
    ]
    \tikzstyle{chentinyattr} = [
      align=center,
      draw=chenattrcolor,
      thick,
      anchor=base,
      text=chenattrcolor,
      shape=ellipse,
      minimum size=3pt,
      opacity=0.95,
      inner sep=2pt,
      font=\ttfamily\footnotesize
    ]
    \tikzstyle{chenmultiattr} = [
      align=center,
      draw=chenattrcolor,
      thick,
      double,
      anchor=base,
      text=chenattrcolor,
      shape=ellipse,
      minimum size=3pt,
      opacity=0.95,
      inner sep=2pt,
      font=\ttfamily\small
    ]
    \tikzstyle{chenderivattr} = [
      align=center,
      draw=chenattrcolor,
      thick,
      dashed,
      anchor=base,
      text=chenattrcolor,
      shape=ellipse,
      minimum size=3pt,
      opacity=0.95,
      inner sep=2pt,
      font=\ttfamily\small
    ]
    \tikzstyle{relat} = [
      align=center,
      thick,
      opacity=0.9,
      draw=chenattrcolor
    ]
    \tikzstyle{relattext} = [
      sloped,
      text=chenattrcolor,
      align=center,
      opacity=0.95,
      above=0.0pt,
      font=\ttfamily\small
    ]
    \tikzstyle{cstep} = [
      fill=cstepcolor,
      anchor=base,
      text=white,
      align=center,
      shape=ellipse,
      minimum size=5pt,
      opacity=0.95,
      inner sep=2pt,
      font=\sffamily\scriptsize\itshape
    ]
    \tikzstyle{flow} = [
      align=center,
      very thick,
      opacity=0.9,
      draw=flowcolor
    ]
    \tikzstyle{flowtext} = [
      sloped,
      text=flowcolor,
      align=center,
      opacity=0.95,
      above=0.0pt,
      font=\sffamily\footnotesize\slshape
    ]
    \tikzstyle{boundaryline} = [
      dashed,
      align=center,
      very thick,
      opacity=0.9,
      draw=boundarycolor
    ]
    \tikzstyle{inviso} = [
      transparent
    ]


% Define a `datastore' shape for use in DFDs.
% This inherits from a rectangle and only draws two
% horizontal lines.
\makeatletter
\pgfdeclareshape{datastore}{
  \inheritsavedanchors[from=rectangle]
  \inheritanchorborder[from=rectangle]
  \inheritanchor[from=rectangle]{center}
  \inheritanchor[from=rectangle]{base}
  \inheritanchor[from=rectangle]{north}
  \inheritanchor[from=rectangle]{north east}
  \inheritanchor[from=rectangle]{east}
  \inheritanchor[from=rectangle]{south east}
  \inheritanchor[from=rectangle]{south}
  \inheritanchor[from=rectangle]{south west}
  \inheritanchor[from=rectangle]{west}
  \inheritanchor[from=rectangle]{north west}
  \backgroundpath{
    %  store lower right in xa/ya and upper right in xb/yb
    \southwest \pgf@xa=\pgf@x \pgf@ya=\pgf@y
    \northeast \pgf@xb=\pgf@x \pgf@yb=\pgf@y
    \pgfpathmoveto{\pgfpoint{\pgf@xa}{\pgf@ya}}
    \pgfpathlineto{\pgfpoint{\pgf@xb}{\pgf@ya}}
    \pgfpathmoveto{\pgfpoint{\pgf@xa}{\pgf@yb}}
    \pgfpathlineto{\pgfpoint{\pgf@xb}{\pgf@yb}}
 }
}
\makeatother
\usepackage{weiwBTree}
\pgfmathdeclarefunction{gauss}{2}{%
  \pgfmathparse{1/(#2*sqrt(2*pi))*exp(-((x-#1)^2)/(2*#2^2))}%
}


\newcommand{\tikzmark}[2]{
    \tikz[overlay,remember picture,baseline]
    \node[anchor=base] (#1) {$#2$};
}


